\documentclass{article}


\author{Manuel Landesfeind\\manuel@gobics.de}
\title{HiVi User Manual\\Version 2.0}

\begin{document}
\maketitle
\tableofcontents




\section{Files distributed with HiVi}
HiVi requires specific database files.

\subsection{HiVi Database}
A database for HiVi is a ZIP file containing exactly three files:
\begin{description}
	\item[info] containing a textual description of the database
	\item[schema] contains the hierarchic data
	\item[mapping] maps given ID against the hierarchy in \texttt{schema}
\end{description}

\paragraph{The \texttt{info} file} This file contains unstructured text that
can be displayed in an information box in the GUI. This file is optional an
has no purpose after all.

\paragraph{The \texttt{schema} file} This file is a structured text file in
CSV format (comma-separated-values). Each line has to contain at least two
cells and might contain as many as wanted. The first cell of earch line is the
accession for this entity in the hierarchy and has to be unique for this
schema. Second cell contains a name or textual description (MUST not contain a
tabulator). All further cells contain the parents of the current entity.

\paragraph{The \texttt{mapping} file} This file maps the gene-ids to the
entities in the schema file. A mapping file is in CSV too, using a tabulator
symbol as separator. Each line has to split up into exactly four parts:
\begin{enumerate}
	\item the gene id
	\item the schema identity it is mapped to
	\item the quality of this assignment (this can be a numerical value of
		whatever kind but has to be consitent within a single database, e.g.
		MUST always be percent or always be pValue)
	\item a textual description (e.g. FASTA header from BLAST results)
\end{enumerate}




\section{Inspecting the results}

